\documentclass[11pt, a4paper]{article}
\usepackage[inner=1in,outer=1in,top=1in,bottom=1in]{geometry}
\pagestyle{empty}
\usepackage{placeins}
\usepackage{graphicx}
\usepackage{fancyhdr, lastpage, bbding, pmboxdraw}
\usepackage[usenames,dvipsnames]{color}
\definecolor{darkblue}{rgb}{0,0,.6}
\definecolor{darkred}{rgb}{.7,0,0}
\definecolor{darkgreen}{rgb}{0,.6,0}
\definecolor{red}{rgb}{.98,0,0}
\usepackage[colorlinks,pagebackref,pdfusetitle,urlcolor=darkblue,citecolor=darkblue,linkcolor=darkred,bookmarksnumbered,plainpages=false]{hyperref}
%\renewcommand{\thefootnote}{\fnsymbol{footnote}}
\usepackage{amsmath}
\usepackage{amssymb}

\pagestyle{fancyplain}
\fancyhf{}
\lhead{ \fancyplain{}{\CourseTitle} }
%\chead{ \fancyplain{}{} }
\rhead{ \fancyplain{}{\CourseSemester \CourseYear} }
%\rfoot{\fancyplain{}{page \thepage\ of \pageref{LastPage}}}
\fancyfoot[RO, LE] {page \thepage\ of \pageref{LastPage} }
\thispagestyle{plain}
\usepackage{tabularx}


%%%%%%%%%%%%%%%%%%%%%%%%%%%%%%%%%%%%
\usepackage{xspace}

\newcommand{\CourseNumber}{NPRE412}
\newcommand{\CourseTitle}{Nuclear Power Economics and Fuel Management\xspace}%
\newcommand{\CourseInstructor}{Prof. Kathryn Huff\xspace}%
\newcommand{\CourseSemester}{Spring\xspace}%
\newcommand{\CourseYear}{2020\xspace}%
\newcommand{\CourseDays}{MWF\xspace}%
\newcommand{\CourseStart}{10:00am\xspace}%
\newcommand{\CourseEnd}{10:50am\xspace}%
\newcommand{\CourseInstructorEmail}{kdhuff@illinois.edu}
\newcommand{\HuffOfficeHourPlace}{118 Talbot Laboratory\xspace}
\newcommand{\CourseRoom}{225A\xspace}%
\newcommand{\CourseBuilding}{Talbot Laboratory\xspace}%
\newcommand{\CourseUniversity}{University of Illinois, Urbana-Champaign\xspace}%
\newcommand{\TeachingAssistant}{Gwendolyn Chee\xspace}%
\newcommand{\TAEmail}{Gwendolyn Chee\xspace}%
\newcommand{\TAOfficeHourDays}{Mondays and Tuesdays\xspace}%
\newcommand{\TAOfficeHourStart}{2pm\xspace}%
\newcommand{\TAOfficeHourEnd}{3pm\xspace}%
\newcommand{\TAOfficeHourPlace}{226 Talbot Laboratory\xspace}
%\newcommand{\Course<++>}{<++>}
%\newcommand{\Course<++>}{<++>}
%%%%%%%%%%%%%%%%%%%%%%%%%%%%%%%%%%%%
\title{\CourseNumber: \CourseTitle\\}
\author{\CourseUniversity}
\date{\CourseSemester \CourseYear}
\begin{document}
\maketitle
%\setlength{\unitlength}{1in}
\renewcommand{\arraystretch}{1.5}
\begin{center}
\begin{table}[h]
\begin{tabularx}{\textwidth}{rXrX}
\hline
\textbf{Instructor:} & \CourseInstructor & \textbf{Time:} & \CourseDays \CourseStart -- \CourseEnd \\
\textbf{Email:} &  \href{mailto:\CourseInstructorEmail}{\CourseInstructorEmail} & \textbf{Place:} & \CourseRoom \CourseBuilding\\
\hline
\end{tabularx}
\end{table}
\end{center}

\paragraph{Course Pages:}
\begin{enumerate}
        \item \url{https://compass2g.illinois.edu}
        \item \url{https://github.com/katyhuff/\CourseNumber}
        \item \url{https://mybinder.org/v2/gh/katyhuff/npre412/master}
\end{enumerate}

\paragraph{TA Office Hours:} The teaching assistant for the course, 
\TeachingAssistant, will hold office hours \TAOfficeHourDays from 
\TAOfficeHourStart to \TAOfficeHourEnd in \TAOfficeHourPlace.



\paragraph{Professor Office Hours:} Prof. Huff will hold office hours by appointment
only, in her office, \HuffOfficeHourPlace. Please make use of the teaching
assistant and your colleagues before booking an appointment with Prof. Huff.
You can make an appointment at \url{katyhuff.youcanbook.me}. Appointments 
must be booked at least 24 hours ahead of time.
If the door to \HuffOfficeHourPlace is open, Prof. Huff may be available for very brief 
questions. In that case, feel free to drop by.

\paragraph{Main References:}
A few essential references for this course will be assigned as readings. The 
recommended text for this course is \cite{tsoulfanidis_nuclear_2013}.
\bibliographystyle{unsrt}
\renewcommand{\refname}{\normalfont\selectfont\normalsize}\vspace{-1cm} 
\bibliography{bibliography}

\paragraph{Objectives:} 

This course will equip students to:

\begin{itemize}
\item Quantify impacts of the nuclear power industry
\item Calculate nuclear fuel cycle and capital costs for thermal and fast reactors.
\item Optimize nuclear fuel management for lowest energy costs and highest system performance.
\item Differentiate among features of fossil fuel systems, fission systems, and controlled thermonuclear fusion systems.
\item Quantiatively analyze nuclear fuel cycle technologies for both once-through and closed strategies.
\item Comparatively assess spent fuel storage, reprocessing, and disposal strategies.
\end{itemize}

\paragraph{Prerequisites:} 
\begin{itemize}
\item Junior standing is encouraged.
\item NPRE 402 or 247
\end{itemize}

\paragraph{Grading Policy:} Grades will be assigned as a weighted sum of the 
following work.

\begin{table}[h]
\begin{tabularx}{\textwidth}{Xrr}
        \textbf{Work} & \textbf{Weight (Undergraduate)} & \textbf{Weight (Graduate)} \\
\hline
\textbf{Quizzes}         & (10\%)   & (10\%)\\
\textbf{Homework}        & (20\%)   & (20\%)\\
\textbf{Midterm}         & (20\%)   & (20\%)\\
\textbf{Long Read}       & (20\%)   & (20\%)\\
\textbf{Project}         & (30\%)   & (30\%)\\
\hline
\textbf{Total}           & (100\%) & (100\%)\\
\end{tabularx}
\end{table}

\paragraph{Important Dates:}
\begin{center} \begin{minipage}{3.8in}
\begin{flushleft}
Midterm      \dotfill 10:00-10:50am, March 13, 2020\\
%Project Deadline \dotfill ~Month Day \\
Project Presentations       \dotfill 7:00-10:00 p.m., May 12, 2020\\
\end{flushleft}
\end{minipage}
\end{center}



\paragraph{Class Policies:}

\begin{itemize}
\item[] \textbf{Integrity:} This is an institution of higher
learning. You will be swiftly ejected from the course if you are caught
undermining its integrity. Note the
\href{http://www.provost.illinois.edu/academicintegrity/students.html}{Student's
Quick Reference Guide to Academic Integrity} and the
\href{http://studentcode.illinois.edu/article1_part4_1-401.html}{Academic
Integrity Policy and Procedure}.
\item[] \textbf{Attendance:} Regular attendance is mandatory. Request approval
        for absence for extenuating circumstances prior to absence.
\item[] \textbf{Electronics:} Active participation is essential and expected.
        Accordingly, students must turn off all electronic devices (laptop,
        tablets, cellphones, etc.) during class. Exceptions may be granted for
        laptops if engaging in computational exercises or taking notes.
\item[] \textbf{Collaboration:} Collaboratively reviewing course materials and
        studying for exams with fellow students can be enriching.  This is
                recommended.  However, unless otherwise instructed, homework
                assignments are to be completed independently and materials
                submitted as homework should be the result of one's own
                independent work.
\item[] \textbf{Late Work:} Late work has a halflife of 1 hour. That is,
        adjusted for lateness, your grade $G(t)$ is a decaying percentage of
                the raw grade $G_0$. An assignment turned in $t$ hours late
                will receive a grade according to the following relation:
\begin{align*}
        G(t) &= G_0e^{-\lambda t}
        \intertext{where}
        G(t) &= \mbox{grade adjusted for lateness}\\
        G_0 &= \mbox{raw grade}\\
        \lambda &= \frac{ln(2)}{t_{\frac{1}{2}}} = \mbox{decay constant} \\
        t &= \mbox{time elapsed since due [hours]}\\
        t_{1/2} &= 1 = \mbox{half-life [hours]} \\
\end{align*}
\item[] \textbf{Make-up Work:} There will be no negotiation about late work
        except in the case of absence documented by an absence letter from the
                Dean of Students.  The university policy for requesting such a
                letter is in
                \href{http://studentcode.illinois.edu/article1_part5_1-501.html}{the
                Student Code}. Please note that such a letter is appropriate
                for many types of conflicts, but that religious conflicts
                require special early handling. In accordance with university
                policy, students seeking an excused absence for religious
                reasons should complete the Request for Accommodation for
                Religious Observances Form, which can be found on the Office of
                the Dean of Students website. The student should submit this
                form to the instructor and the Office of the Dean of Students
                by the end of the second week of the course to which it
                applies.

\item[] \textbf{Grade Disputes:} It is important that you understand and agree
        with the grade you receive on assignments and exams. If you would like
        to dispute your score, you must send an explanation by email to Prof.
        Huff within one week of recieving the grade.
        \textbf{Do not expect me to regrade anything while in conversation with
        you} as that would not be fair to the other students in the class, whose
        homeworks were graded without them present.  If you request a regrade,
        be aware that the entire assignment will be regraded and is subject to
        double-jeopardy: it is possible that your score will go down.
        Regrade requests should be based on an error on my part (e.g., adding
        up the points incorrectly) or what you suspect is a misunderstanding of
        your work (e.g., arriving at the correct answer using an unexpected
        technique). Regrade requests that argue with the rubric (e.g., ``this is
        wrong, but you took too many points off'') will be returned without
        consideration.
        \textbf{Your work should stand alone.} If an assignment is disorganized or
        ambiguous, and requires an extensive explanation to the grader, you
        will likely still lose points. The homeworks not only evaluate your
        understanding of the material - they also evaluate your ability to
        communicate that understanding clearly and concisely.

\end{itemize}

\paragraph{Accessibility:} I hope that this course will be inclusive and
accommodating for all learners. As such, I am committed upholding the vision
and values of \href{http://www.inclusiveillinois.illinois.edu/index.html}{Inclusive Illinois}
in my
classroom.  With regard to accommodating all learners, please note that many
resources are provided through
\href{http://disability.illinois.edu/academic-support/accommodations}{the
Division of Disability Resources and Educational Services}.  To request
particular accommodations, please contact me as soon as possible so that we can
work out any necessary arrangements.

\paragraph{Other Resources:}
University students typically experience a wide range of stressors during their
time on campus. Accordingly, campus resources exist to help students manage
stress levels, mental health, physical health, and emergencies while navigating
this environment. I hope you will take advantage of these campus resources as
soon as they can be of help.

\begin{itemize}
\item \href{https://campusrec.illinois.edu/}{The Campus Recreational Centers}
\item \href{http://counselingcenter.illinois.edu/}{The Counselling Center}
\item \href{https://mckinley.illinois.edu/}{The McKinley Health Clinic}
\item \href{http://www.mckinley.illinois.edu/medical-services/mental-health}{The McKinley Mental Health Clinic}
\item \href{https://odos.illinois.edu/community-of-care/emergency-dean/}{The Emergency Dean}
\end{itemize}

\paragraph{Run. Hide. Fight.}
It is important that we take time to prepare for a situation in which our
safety could depend on our ability to react quickly. Please review the
university guidance on responding to emergency situations
\url{https://police.illinois.edu/emergency-preparedness/run-hide-fight/}.
Take a moment to learn the different ways to leave
this building. If there’s ever a fire alarm or something like that, you’ll know
how to get out and you’ll be able to help others get out. Next, figure out the
best place to go in case of severe weather - we’ll need to go to a low-level in
the middle of the building, away from windows. And finally, if there’s ever
someone trying to hurt us, our best option is to run out of the building. If we
cannot do that safely, we’ll want to hide somewhere we can’t be seen, and we’ll
have to lock or barricade the door if possible and be as quiet as we can. We
will not leave that safe area until we get an Illini-Alert confirming that it’s
safe to do so. If we can’t run or hide, we’ll fight back with whatever we can
get our hands on. If you want to better prepare yourself for any of these
situations, visit \url{police.illinois.edu/safe}. Remember you can sign up for
emergency text messages at \url{emergency.illinois.edu}.



\pagebreak
\FloatBarrier
\renewcommand{\arraystretch}{1}
\begin{table}[h]
\begin{center}
\begin{tabular}{lllcllll}
\multicolumn{8}{c}{\textbf{Course Schedule:}\textit{ Note that this schedule is subject to change}}\\
&&&&&&&\\
\textbf{Date} & \textbf{Week} & \textbf{Day} & \textbf{Unit} & \textbf{Chap.} & \textbf{Quiz} & \textbf{HW} & \textbf{HW}\\
 &  &  &  &  &                                                                                          & \textbf{Given} & \textbf{Due}\\
\hline
\hline
01-22 & 1 & W & Intro      & 1 &           &      &    \\
01-24 & 1 & F & Overview   & 1 &           &  HW1 &    \\
01-27 & 2 & M & Overview   & 1 &        Q1 &      &    \\
01-29 & 2 & W & Economics  & 8 &           &      &    \\
01-31 & 2 & F & Economics  & 8 &           &  HW2 & HW1\\
02-03 & 3 & M & Economics  & 8 &        Q2 &      &    \\
02-05 & 3 & W & Economics  & 8 &           &      &    \\
02-07 & 3 & F & Economics  & 8 &           &  HW3 & HW2\\
02-10 & 4 & M & Mining \& Milling & 2 & Q3 &      &    \\
02-12 & 4 & W & Mining \& Milling & 2 &    &      &    \\
02-14 & 4 & F & Mining \& Milling & 2 &    &  HW4 & HW3\\
02-17 & 5 & M & Conversion & 3 &        Q4 &      &    \\
02-19 & 5 & W & Enrichment & 3 &           &      &    \\
02-21 & 5 & F & Enrichment & 3 &           &  HW5 & HW4\\
02-24 & 6 & M & Enrichment & 3 &        Q5 &      &    \\
02-26 & 6 & W & Enrichment & 3 &           &      &    \\
02-28 & 6 & F & Fuel Fabrication & 4 &     &  HW6 & HW5\\
03-02 & 7 & M & Fuel Fabrication & 4 &  Q6 &      &    \\
03-04 & 7 & W & Reactors & 5 &             &      &    \\
03-06 & 7 & F & Reactors & 5 &             &  HW7 & HW6\\
03-09 & 8 & M & Reactors & 5 &          Q7 &      &    \\
03-11 & 8 & W & Reactors & 5 &             &      &    \\
03-13 & 8 & F & \textbullet~\textbf{Midterm} \textbullet &  &  &  HW8 & HW7\\
03-16 & 9 & M & \textbullet~\textbf{No Class} \textbullet &  &  &  & \\
03-18 & 9 & W & \textbullet~\textbf{No Class} \textbullet &  &  &  & \\
03-20 & 9 & F & \textbullet~\textbf{No Class} \textbullet &  &  &  & \\
03-23 & 10 & M & Fuel In-Core & 6 &    Q9  &       & \\
03-25 & 10 & W & Fuel In-Core & 6 &        &       & \\
03-27 & 10 & F & Reprocessing & 7 &        &   HW9 & HW8\\
03-30 & 11 & M & Reprocessing & 7 &    Q10 &       & \\
04-01 & 11 & W & Reprocessing & 7 &        &       & \\
04-03 & 11 & F & Reprocessing & 7 &        &  HW10 & HW9\\
04-06 & 12 & M & Reprocessing & 7 &    Q11 &       & \\
04-08 & 12 & W & HLW & 9 &                 &       & \\
04-10 & 12 & F & HLW & 9 &                 &  HW11 & HW10\\
04-13 & 13 & M & HLW & 9 &             Q12 &       & \\
04-15 & 13 & W & HLW & 9 &                 &       & \\
04-17 & 13 & F & HLW & 9 &                 &  HW12 & HW11\\
04-20 & 14 & M & LLW & 10 &             Q13 &       & \\
04-22 & 14 & W & LLW & 10 &                &       & \\
04-24 & 14 & F & Nonproliveration & 11 &   &  HW13 & HW12\\
04-27 & 15 & M & Nonproliferation & 11 & Q14 &     & \\
04-29 & 15 & W & Environment & 12 &        &       & \\
05-01 & 15 & F & Environment & 12 &        &       & HW13\\
05-04 & 16 & M & Environment & 12 &    Q15 &       & \\
05-06 & 16 & W & Environment & 12 &        &       & \\
05-12 & 17 & T & \textbullet~\textbf{Presentations} \textbullet &  &  &  & \\
\end{tabular}
\end{center}
\end{table}
\FloatBarrier



%%%%%% THE END 
\end{document} 
