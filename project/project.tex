% use the answers clause to get answers to print; otherwise leave it out.
\documentclass[12pts, answers]{exam}
%\documentclass[12pts]{exam}
\RequirePackage{amssymb, amsfonts, amsmath, latexsym, verbatim, xspace, setspace}

% By default LaTeX uses large margins.  This doesn't work well on exams; problems
% end up in the "middle" of the page, reducing the amount of space for students
% to work on them.
\usepackage[margin=1in]{geometry}
\usepackage{enumerate}
\usepackage{hyperref}

% Here's where you edit the Class, Exam, Date, etc.
\newcommand{\class}{NPRE 412}
\newcommand{\term}{Spring 2021}
\newcommand{\assignment}{Project}
\newcommand{\duedate}{2021.05.14}
%\newcommand{\timelimit}{50 Minutes}

\newcommand{\nth}{n\ensuremath{^{\text{th}}} }
\newcommand{\ve}[1]{\ensuremath{\mathbf{#1}}}
\newcommand{\Macro}{\ensuremath{\Sigma}}
\newcommand{\vOmega}{\ensuremath{\hat{\Omega}}}

% For an exam, single spacing is most appropriate
\singlespacing
% \onehalfspacing
% \doublespacing

% For an exam, we generally want to turn off paragraph indentation
\parindent 0ex

%\unframedsolutions
\usepackage{bibentry}
\begin{document} 

% These commands set up the running header on the top of the exam pages
\pagestyle{head}
\firstpageheader{}{}{}
\runningheader{\class}{\assignment\ - Page \thepage\ of \numpages}{Due \duedate}
\runningheadrule

\class \hfill \term \\
\assignment \hfill Due \duedate\\
%\begin{flushright}
%\begin{tabular}{p{5in} r l}
%\end{tabular}
%\end{flushright}
\rule[1ex]{\textwidth}{.1pt}

%%%%%%%%%%%%%%%%%%%%%%%%%%%%%%%%%%%%%%%%%%%%%%%%%%%%%%%%%%%%%%%%%%%%%%%%%%%%%%%%%%%%%
%%%%%%%%%%%%%%%%%%%%%%%%%%%%%%%%%%%%%%%%%%%%%%%%%%%%%%%%%%%%%%%%%%%%%%%%%%%%%%%%%%%%%

In NPRE 412, a large part of your grade is earned 
with a project proposal and a final project.  This project is intended to tie together the lessons of the course
with an independent analysis of a relevant topic in the context of nuclear
power economics and fuel management (or, the nuclear fuel cycle). The project
will be assessed as independent research work, much like a journal article
undergoes peer review. I will be looking for :

\begin{itemize}
\item Relevance
\item Novelty
\item Technical Detail
\item Analytic Rigor
\item Verifiability
\item Clarity
\item A Conclusion
\end{itemize}

This work will consist of two deliverables, a proposal and a final report.


% ---------------------------------------------
\begin{questions}
\addpoints
% intro
\question[10] \textbf{Proposal: Due 2021.03.10}

To help determine of reasonable scope, the first step of the project will be a
proposal. Once you submit this proposal, I will respond with feedback. Much
like a conference abstract, the proposal should meet the following guidelines:

\begin{itemize}
\item Minimum 500 words.
\item Maximum 1000 words.
\item Two columns.
\item Reasonable margins.
\item 10 pt font or larger.
\item State the question you plan to answer.
\item Summarize the current state of the art in the literature.
\item Motivate the problem, explaining its relevance.
\item Describe the approach and methods you will take to answer the question.
\item Propose an outline of the analysis, software, data, and/or conclusions that will be delivered.
\end{itemize}

Feel free to run ideas past me as needed via email. I would be happy to provide 
        specific feedback on a draft of your proposal (once) before it is due. 
        For your guidance, a list of example topics appears at the end of this 
        document. Feel free to choose one of these.  However, choosing a 
        creative topic of your choice is also encouraged.


\question[90] \textbf{Final Report: Due \duedate}

Prepare a final document in the style of a journal article or conference
proceedings. It should meet the following guidelines:

\begin{itemize}
\item Minimum 3000 words.
\item Maximum 10000 words.
\item Two columns.
\item Reasonable margins.
\item 10 pt font or larger.
\item State the question you answered.
\item Comprehensively report and cite the current state of the art in the literature.
\item Motivate the problem, explaining its relevance.
\item Describe the approach, methods, and other elements of your solution.
\item Describe in detail: the analysis, software, data, conclusions produced in this work.
\item Include publication quality graphs and figures.
\item Cite and provide data and code generated for this work sufficient to reproduce the conclusions.
\item Compare this result to previous results in the literature, reinforce the relevance of the work.
\item Suggest future work.
\end{itemize}

\end{questions}


\section*{Topic Examples}
For your guidance, a list of example topics appears here.
Feel free to choose one of these.  However, choosing a creative topic
of your choice is encouraged. This section may be expanded if I have new ideas
as the semester progresses.

\paragraph{Economic and Carbon Impacts of Potential Illinois Nuclear Plant Closures} 
11 carbon free nuclear power reactors at 6 sites produce the majority of electricity in Illinois
and critically underpin its clean energy future. \emph{Quantitatively} demonstrate the
role nuclear energy could play in maximizing job creation, minimizing cost, and meeting Illinois’
carbon goals through 2050. Review the policy recommendations in the Clean 
Energy Jobs Act and consider the timeline of possible Illinois Nuclear Plant 
closures currently in the news. Conduct a 50-year techno-economic optimization of
the Illinois energy system and to analyze scenarios with and without the 
current at-risk plants to
compare and contrast the economic and carbon implications of these energy 
futures. You should expect to reveal regionally relevant findings consistent with the February 2021 National
Academy of Sciences, Engineering, and Medicine report, ``Accelerating Decarbonization of the
U.S. Energy System'', which determined unequivocally that US decarbonization will require
keeping existing nuclear plants open. Consider tools like Temoa, which may be 
helpful in this endeavor.

\paragraph{Quantifying the Costs of Weatherizing the Texas Grid}
Using any references you can find along with the 2011 report on weatherization needs for the ERCOT grid and data from 
EIA regarding the Texas blackouts in 2021, quantitatively establish the 
following as accurately as is feasible: 
\begin{itemize}
        \item The cost (to the TX ratepayers) of weatherization actions taken by TX generators between 2011 
                and 2021.
        \item The total weatherization costs (to the TX ratepayers) that would have been incurred had all 
                of the 2021 recommendations been employed.
        \item The total cost (to the TX ratepayers) of the blackouts. Include 
                housing damage estimates, the cost of loss of life, loss of 
                work, etc. 
        \item Consider the time value of money and compare these scenarios. 
                Assume that taking all weatherization actions 10 years ago would have 
                avoided all  February 2021 damages. Would paying for this preparation (10 years ago) have 
                cost the ratepayer more or less than the ultimate damages 
                ultimately cost?
\end{itemize}

\paragraph{Impact of a Zero Emissions Tax Credit} New York state recently
(summer 2016) implemented a zero emissions tax credit. It was enough to save
the Fitzpatrick nuclear generating station. Were the parameters of that
state-level to be implemented accross the US, either federally or individually
in each state, the risk of owning and building nuclear plants would decrease.
Quantify the change in risk. Predict the impacts to the nuclear industry. Would
it be enough to save at-risk plants? Would we likely see an increase in new
builds? What other impacts might we see?

\paragraph{The Likelihood and Implications of the Duck Curve} The California
Independent System Operators published the ``duck chart.'' This curve,
describing the predicted mid-day overgeneration of grid-bound electricity,
caused by installations of solar, primarily, makes load-following generation
sources or storage methods necessary. A few questions that would make
interesting projects on this topic include:
\begin{itemize}
\item What is the level of alarm appropriate in reaction to this chart and why?
\item Can you suggest a novel strategy that would allow nuclear generation to
be adapted to load follow?
\item What would the impact to nuclear power be if this situation is allowed to
proceed and there is no curtailment of variable generation sources?
\item How could current storage technologies smooth this curve?
\end{itemize}

\paragraph{Liquid vs. Solid Fuelled Molten Salt Reactor Source Term and Release
Pathways} Reactor designs involving molten salts can have either solid fuel or
fluidized fuel. In the fluid fuel case, proponents are often heard to say they
can't melt down because they operate safely at a melted state. Compare the
source term and release pathways of a containment breach in a solid fuelled
salt reactor vs. a fluid fuelled one.

\paragraph{Economics of Uranium Extraction of Seawater} A great deal of
research is being undertaken to lower the costs of uranium extraction from
seawater. Can you replicate the results of a previous calculation of those
costs? To what extent are those results sensitive to uncertain assumptions
(assumptions of a political, technical, or economic nature, it matters not.)

\paragraph{Metrics for Proliferation in the Nuclear Fuel Cycle} Suggest a
metric that captures proliferation concern in a nuclear fuel cycle scenario.

\paragraph{Assessment of Dose To Workers in Reprocessing Schemes} Propose a model for
calculating the dose impact on workers within an arbitrary fuel cycle. Include
dose due to all steps of the fuel cycle, including reprocessing and disposal.
Compare and contrast fuel cycle strategies with and without reprocessing.

\paragraph{Fuel Cycle Transition Scenario} Choose one of the Evaluation Groups
identified by the Fuel Cycle Options Evaluation and Screening. Use a simulator
(e.g. Cyclus, CLASS, or Orion) or your own model to assess the
time-to-transition from our current reactor fleet to 100\% deployment of the
new technology (e.g. SFRs, MSRs, etc.)

\paragraph{Repository Cost Estimation} Suggest a high level waste repository
site. Suggest appropriate waste forms, waste packages, and other disposal
system design features for this geology. Estimate construction rates, loading
rates, transportation costs, and a closing timeline. With these estimates,
defended by analysis, conduct a life cycle cost estimation of your proposed
site.

%\bibliographystyle{plain}
%\bibliography{}

\end{document}

